\documentclass[11pt]{article} 

% packages with special commands
\usepackage{amssymb, amsmath}
\usepackage{epsfig}
\usepackage{array}
\usepackage{ifthen}
\usepackage{color}
\usepackage{fancyhdr}
\usepackage{graphicx}
%\usepackage{mathtools}
\definecolor{grey}{rgb}{0.5,0.5,0.5}

\begin{document}
\newcommand{\tr}{\text{tr}}
\newcommand{\E}{\textbf{E}}
\newcommand{\diag}{\text{diag}}
\newcommand{\argmax}{\text{argmax}}
\newcommand{\argmin}{\text{argmin}}
\newcommand{\Cov}{\text{Cov}}
\newcommand{\Vol}{\text{Vol}}
\pagestyle{fancy}

\title{Removing noise-induced bias in diffusion MRI images}

\author{Charles Zheng, Franco Pestilli, Yuval Benjamini, Robert Dougherty, Hua Wu, and Ariel Rokem}

\maketitle

\begin{abstract}
Diffusion-weighted MR imaging (DWI) is the only method we currently
have to measure connections between different parts of the human brain
\emph{in vivo}.  To elucidate the structure of these connections,
algorithms for tracking bundles of axonal fibers through the
subcortical white matter rely on local estimates of the \emph{fiber
  orientation distribution function} (fODF) in different parts of the
brain. These functions describe the relative abundance of populations
of axonal fibers crossing each other in each location.  The accuracy
of the reconstruction is limited due to interference from various
sources of noise, including physiological processes, motion of the
subject's head, heat in the MRI coil, and eddy currents.  Although
scanning protocols already include preprocessing steps to minimize the
effects of motion and eddy currents, artifacts from motion correction
are still visible in the preprocessed data, and furthermore,
preprocessing protocols do not yet address how to correct for
physiological and thermal noise.  We present a unified model for all
sources of noise in diffusion MRI data, which partitions noise into
two categories: \emph{local noise}, which includes noise sources like
thermal noise which are not highly correlated between voxels, and
\emph{non-local noise}, which includes noise sources such as
physiological noise and motion correction artifacts, which affect
multiple nearby voxels.  Both types of noise induce \emph{bias} in the
diffusion MRI signal: meaning that the average intensity over many
repeated observations is systematically larger than the intensity as
would be obtained from a noise-free setting.  This bias is problematic
for existing reconstruction algorithms such as CSD and Bedpost-X which
are derived under an assumption of \emph{unbiased} Gaussian noise.  We
develop an algorithm for removing the bias from both local and
non-local noise. Local noise can be removed in a straightforward
manner by using soft-thresholding. Optimally removing non-local noise
is more difficult because it requires pooling information from
neighboring voxels: we use K-means clustering to partition the voxels
and model the non-local noise within each partition via factor
analysis.  We also estimate the fraction of noise contributed by each
noise source to the preprocessed data by comparing the combined-coil
MRI image to the MRI images from each of the 32 coils in our data.  We
validate our denoising procedure via two different metrics: prediction
error on a replicate dataset, and reliability of the parameter
estimates.  Since the parameter estimates have a variable number of
fiber directions, we employ the earth mover's distance metric to
measure reliability.
\end{abstract}

\section{Introduction}


\section{References}

Le Bihan D, Mangin JF, Poupon C, Clark CA, Pappata
S, Molko N, Chabriat H. (2001). Diffusion tensor imaging:
concepts and applications. \emph{Journal of magnetic resonance imaging},
13(4), 534-546.

Tournier J-D, Calamante F, Connelly A (2007). Robust determination of the
fibre orientation distribution in diffusion MRI: non-negativity constrained
super-resolved spherical deconvolution. {\it Neuroimage} 35:1459–72

Tournier, J.-D., Yeh, C.-H., Calamante, F., Cho, K.-H., Connelly, A., and
Lin, C.-P. (2008). Resolving crossing fibres using constrained spherical
deconvolution: validation using diffusion-weighted imaging phantom
data. NeuroImage, 42: 617–25.

Basser PJ. Quantifying errors in fiber-tract direction and diffusion tensor
field maps resulting from MR noise. Proc. Int. Soc. Magn. Reson. Med. 1997

Aganj I, Lenglet C, Jahanshad N, Yacoub E, Harel N, Thompson PM,
Sapiro G. (2011). A Hough transform global probabilistic approach to
multiple-subject diffusion MRI tractography. \emph{Medical image
  analysis}, 15(4), 414-425.

Frank L. Anisotropy in high angular resolution diffusion-weighted
MRI.  \emph{Magnetic Resonance in Medicine} Volume 45, Issue 6, pages
935–939, June 2001

Dell’Acqua F, Rizzo G, Scifo P, Clarke RA, Scotti G, Fazio F (2007). A
model-based deconvolution approach to solve fiber crossing in
diffusion-weighted MR imaging. {\it IEEE Trans Biomed Eng} 54:462–72

Behrens TEJ, Berg HJ, Jbabdi S, Rushworth MFS, and Woolrich MW (2007).  Probabilistic
diffusion tractography with multiple fiber orientations: What can we
gain?  {\it NeuroImage} (34):144-45.

Gudbjartsson, H., and Patz, S. (1995). The Rician distribution of noisy MR
data. {\it Magn Reson Med}. 34: 910–914.

Rokem A, Yeatman J, Pestilli F, Kay K, Mezer A, van der Welt S,
Wandell B.  (2013). Evaluating the accuracy of models of diffusion MRI
in white matter.  Submitted.

\end{document}
