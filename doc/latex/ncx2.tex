\documentclass[11pt]{article} 

% packages with special commands
\usepackage{amssymb, amsmath}
\usepackage{epsfig}
\usepackage{array}
\usepackage{ifthen}
\usepackage{color}
\usepackage{fancyhdr}
\usepackage{graphicx}
%\usepackage{mathtools}
\definecolor{grey}{rgb}{0.5,0.5,0.5}

\begin{document}
\newcommand{\tr}{\text{tr}}
\newcommand{\E}{\textbf{E}}
\newcommand{\diag}{\text{diag}}
\newcommand{\argmax}{\text{argmax}}
\newcommand{\Cov}{\text{Cov}}
\pagestyle{fancy}

\title{Working with noncentral chi distributions}

\author{Charles Zheng}

\maketitle

\begin{abstract}
We observe $y_i \sim \chi^2_n(\mu_i^2)$ where $\mu_i = X_i \beta$ for some matrix $X$
and coefficients $\beta$.
Computing the log-likelihood and its derivatives with respect to $\mu$ requires special care.
\end{abstract}

\section{Preliminaries}

Recall the definition of the noncentral $\chi^2$ distribution with $n$ degrees of freedom and noncentrality parameter $\lambda$.
The density is
\[
\sum_{k=0}^\infty \frac{e^{-\lambda/2} (\lambda/2)^k }{k!} \exp[k \log(\lambda/2) + (k + (n/2))\log 2 - \log \Gamma(k + (n/2)) + (k+(n/2)-1) \log x -(x/2)]
\]
A compact way to express the log-likelihood $l$ and its derivatives with respect to the noncentrality parameter
$\dot{l} = (\partial l/\partial \lambda)$, $\ddot{l} = (\partial^2 l/\partial \lambda^2)$ is as follows.

Recall the log-sum-exp function $lse(x_1,\hdots,) = \log \sum_{k=1}^\infty e^{x_k}$.
Define $c = -(\lambda + x)/2 + (n/2)\log 2 + ((n/2)-1)\log x$.
Define $t_k = -\log\Gamma(k + (n/2)) + k \log(\lambda x/4) - \log\Gamma(k+1)$ for $k=0,\hdots$.
Then let $s_0 = c + lse(t_k)$, $s_1 = c + lse(t_k + \log k)$, $s_2 = c + lse(t_k + 2\log k)$.
Then $l = s_0$, $\dot{l} = \lambda^{-1} e^{s_1} - (1/2)e^{s_0}$,and
\[
\ddot{l} = \lambda^{-2} e^{s_2} + (1/4)e^{s_0} - (\lambda^{-2} + \lambda^{-1}) e^{s_1}
\]

Defining $\mu =\sqrt{\lambda}$ and $\ell(\mu) = l(n,\lambda,x)$, we have $\ell = l$, $\dot{\ell} = 2\mu \dot{\ell}$, $\ddot{\ell} = 2\dot{l} + 4\mu^2 \ddot{l}$.

\section{Methods}

The difficulty of the above computations is due to the computation of quantities $s_0,$, $s_1$, $s_2$, which have in common the expression
\[
g(a,b,c,d) := \log\left(\sum_{k=0}^\infty \exp[-\log \Gamma(k + a) - \log \Gamma(k + b) + c\log k + dk]\right)
\]
for some $a, b, c \geq 0$ and $c \in \mathbb{R}$.

\end{document}

